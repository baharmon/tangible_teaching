% !TEX encoding = UTF-8 Unicode

%%
%% Copyright 2007, 2008, 2009 Elsevier Ltd
%%
%% This file is part of the 'Elsarticle Bundle'.
%% ---------------------------------------------
%%
%% It may be distributed under the conditions of the LaTeX Project Public
%% License, either version 1.2 of this license or (at your option) any
%% later version.  The latest version of this license is in
%%    http://www.latex-project.org/lppl.txt
%% and version 1.2 or later is part of all distributions of LaTeX
%% version 1999/12/01 or later.
%%
%% The list of all files belonging to the 'Elsarticle Bundle' is
%% given in the file `manifest.txt'.
%%
%% Template article for Elsevier's document class `elsarticle'
%% with numbered style bibliographic references
%% SP 2008/03/01
%%
%% $Id: elsarticle-template-num.tex 4 2009-10-24 08:22:58Z rishi $
%%
%% \documentclass[preprint,12pt,3p]{elsarticle}
%% \documentclass[preprint,review,12pt]{elsarticle}
%% \documentclass[final,1p,times]{elsarticle}
%% \documentclass[final,1p,times,twocolumn]{elsarticle}
%% \documentclass[final,3p,times]{elsarticle}
%% \documentclass[final,5p,times]{elsarticle}
%% \documentclass[final,5p,times,twocolumn]{elsarticle}

\documentclass[final,3p,times,twocolumn]{elsarticle}

% Packages
\usepackage{graphicx}
\usepackage[utf8]{inputenc}
\usepackage{textcomp,marvosym}
\usepackage{caption}
\usepackage{subcaption}
\usepackage{hyperref}
\usepackage[super]{nth}
\usepackage[inline]{enumitem}
%\usepackage{moreenum}
\usepackage{tabulary}
\usepackage{tabu}
\usepackage{booktabs}
\usepackage{array}
\usepackage[super]{nth}
\usepackage{listings}
\usepackage{float}
\usepackage{upquote}
\usepackage{minted}
\usemintedstyle{bw}
\newcommand{\ra}[1]{\renewcommand{\arraystretch}{#1}}

% Special characters
\usepackage{gensymb}
\usepackage{amsmath,amssymb}
\usepackage{pifont}
\newcommand{\cmark}{\ding{51}}
\newcommand{\xmark}{\ding{55}}

\journal{Landscape and Urban Planning}

\begin{document}

\begin{frontmatter}

\title{Tangible Landscape: a hands-on method for teaching grading and geomorphology}

\author[cga,la]{Brendan Alexander Harmon\corref{cor1}}
\cortext[cor1]{Corresponding author}

\ead{brendan.harmon@gmail.com}
\ead[url]{baharmon@github.io}

\author[cga,meas]{Anna Petrasova}
%\ead{akratoc@ncsu.edu }

\author[cga,meas]{Vaclav Petras}
%\ead{vpetras@ncsu.edu}

\author[cga,la]{Payam Tabrizian}
%\ead{ptabriz@ncsu.edu}

\author[cga,meas]{Helena Mitasova}
%\ead{hmitaso@ncsu.edu}

\author[la]{Andrew Fox}
%\ead{aafox@ncsu.edu}

\author[la]{Carla Delcambre}

\author[la]{Robbie Layton}

\author[la]{Eugene Bressler}

\author[cga,fer]{Ross Meentemeyer}
%\ead{rkmeente@ncsu.edu }

\address[cga]{Center for Geospatial Analytics, North Carolina State University, Raleigh, North Carolina, United States of America}
\address[la]{Department of Landscape Architecture, North Carolina State University, Raleigh, North Carolina, United States of America}
\address[meas]{Department of Marine, Earth, and Atmospheric Sciences, North Carolina State University, Raleigh, North Carolina, United States of America}
\address[fer]{Department of Forestry and Environmental Resources, North Carolina State University, Raleigh, North Carolina, United States of America}

% -------------- ABSTRACT --------------

\begin{abstract}
% overview
We present a hands-on method for teaching grading, geomorphology, and hydrology 
using Tangible Landscape -- a tangible interface for geospatial modeling. 
% tangible landscape
Tangible Landscape couples a physical and digital model of a landscape 
through a real-time cycle 
of hands-on modeling, 3D scanning, geospatial computation, and projection.
With Tangible Landscape students
can sculpt a topographic model of a landscape with their hands
and immediately see how they are changing geospatial analytics like
contours, hill-shading, and flow water. 
% embodied cognition
By kinaesthetically feeling and manipulating the shape of the topography 
students can intuitively learn about 3D topographic form, 
about topographic representations, and how topography controls physical processes.
% experiment
A series of experiments has demonstrated that Tangible Landscape 
is highly intuitive and 
can improve users 3D spatial performance
by enabling embodied cognition.
% teaching methods 
In this paper we propose tangible teaching methods 
using real-time geospatial analytics such as 
contour modeling, simulated water flow, landform recognition, 
cut-fill analysis, and landscape evolution. 
We also discuss how these methods can be integrated into 
landscape architecture and geomorphology curriculums. % include Karl Wegmann?
\end{abstract}

\begin{keyword}
topography \sep grading \sep geomorphology \sep tangible user interfaces \sep geospatial modeling \sep education \sep embodied cognition \sep serious gaming
\end{keyword}

\end{frontmatter}

% -------------- TOC --------------
\tableofcontents
\vfil
\pagebreak

% -------------- BODY --------------

\section{Introduction}\label{intro}
\subsection{Why and how grading is taught}
\noindent
Understanding and representing the earth's surface \\
Cartography \\
Earthmoving \\
In landscape architecture and civil engineering: \\
Focus on contours for representation \\
Focus on meshes for computation and construction \\

\noindent
Teaching grading \\
Represent and manipulate topography \\
Drawing 2D contours \\
Cutting 3D contour models \\

\subsection{Literature review}

\subsubsection{Curriculum review}

\subsubsection{Methods review}

\subsubsection{Spatial cognition}

\subsection{Aims and objectives}
\noindent
% The aim of this research was to
\textbf{Aim:}
Develop an intuitive method for learning about and manipulating
topography, geomorphology, and hydrology
that encourages embodied cognition and metacognition. \\

% form, volume, morphology, processes, scale, and representation

%\noindent
%\textbf{Objectives:}
%\begin{itemize}
%\item Develop an intuitive method for learning about and manipulating
%topography, geomorphology, and hydrology.
%\item Integrate this method into 
%landscape architecture and geomorphology curriculums.
%\end{itemize}

\subsection{Concept}
\noindent
Tangible Landscape as an educational tool\\
3D rendering and VR \\
Serious gaming \\

\section{Tangible Landscape}\label{tangible_landscape}
\subsection{Technology}

\subsection{Experiment}
\cite{Harmon2016c}

\section{Teaching methods}\label{methods}

%\subsection{Contours}

%\subsection{Flooding}

\subsection{Water flow}
\noindent
\textbf{Challenge:} find the highest point from which water will flow to a given point \\
\textbf{Analysis:} r.drain / drain direction \\
\textbf{Interactions:} marker  \\
\textbf{Learning:} water flow \\

\noindent
\textbf{Challenge:} sculpt topography to make water flow to a given point \\
\textbf{Analysis:} water flow \\
\textbf{Interactions:} sculpting  \\
\textbf{Learning:} water flow \\

\noindent
\textbf{Challenge:} sculpt topography around exposed bedrock to make water flow to a given point \\
\textbf{Analysis:} water flow with exposed bedrock \\
\textbf{Interactions:} sculpting  \\
\textbf{Learning:} water flow, volume \\

\noindent
\textbf{Challenge:} given landcover sculpt the topography to make water flow to a designated point \\
\textbf{Analysis:} water flow with landcover \\
\textbf{Interactions:} sculpting  \\
\textbf{Learning:} water flow, volume \\

\subsection{Landforms}
\noindent
\textbf{Challenge:} each round build the specified landforms until the time limit \\
\textbf{Rounds:} 
\begin{enumerate*}
\item build a ridge,
\item build a valley,
\item \ldots
\end{enumerate*}\\
\textbf{Analysis:} geomorphons, contours \\
\textbf{Interactions:} sculpting  and section cut (Blender) \\
\textbf{Learning:} landform types, transitions, and relationships. \\ 

%\noindent
%\textbf{Landscape:} mountainous \\
%\textbf{Analysis:} geomorphons \\
%\textbf{Interaction:} change geomorphons' parameters \\
%\textbf{Learning:} topographic scale invariance \\

\subsection{Cut-fill} % Difference
\noindent
\textbf{Challenge:} build a given landscape \\ % from an initial landscape
\textbf{Analysis:} difference, contours \\
\textbf{Interactions:} sculpting  \\
\textbf{Learning:} volume and cut/fill \\

\subsection{Fabrication}
\noindent
\textbf{Challenge:} digitally 3D model a landscape, %of your own design?
render in 3D, and 3D print, cast a sand model for TL, and scan the sand model.\\
\textbf{Learning:} 

\subsection{Evolution}
\noindent
\textbf{Challenge:} predict and model how a given landscape will evolve \\
\textbf{Analysis:} contours, water flow, erosion-deposition, landscape evolution, and difference \\
\textbf{Interactions:} sculpting  \\
\textbf{Learning:} water flow, erosion-deposition, and landscape evolution \\

\subsection{Trails}
\noindent
\textbf{Challenge:} find the easiest route across the landscape\\
\textbf{Analysis:} trail slope, trail profile, least cost path, r.walk \\
\textbf{Interactions:} markers  \\
\textbf{Learning:} wayfinding \\

\noindent
\textbf{Challenge:} find the easiest route to somewhere with a view of a given feature\\
\textbf{Analysis:} trail slope, trail profile, least cost path, r.walk, viewshed \\
\textbf{Interactions:} markers  \\
\textbf{Learning:} wayfinding and viewsheds \\

\subsection{Design}
% putting it all together
\noindent
\textbf{Challenge:} grade topography and plant trees, shrubs, and groundcover
to minimize erosion and make a beautiful landscape. \\
\textbf{Analysis:} water flow, erosion-deposition, 3D rendering, and VR \\
\textbf{Interactions:} sculpting, planting model trees, and placing felt groundcover  \\
\textbf{Learning:} design thinking, water flow, and erosion-deposition \\

\noindent
\textbf{Challenge:} place a house, make a lake in view of the house, build an access road, build a trail to the lake, fill the lake with a stream, and use all cut-fill. \\
\textbf{Analysis:} viewshed, water flow, ponding, difference, least cost path, r.walk, 3D rendering, and VR \\
\textbf{Interactions:} sculpting, markers, and objects  \\
\textbf{Learning:} design thinking, viewshed, wayfinding, water flow, grading, and cut-fill \\

\subsection{Detailed design}
\noindent
\textbf{Challenge:} Grade a road that stays dry with minimal cut-fill \\ % control storm water runoff
\textbf{Analysis:} v.civil.road \\
\textbf{Interactions:} sculpting  \\
\textbf{Learning:} grading details, water flow, and cut-fill\\

\section{Results}\label{results}
%\subsection{}

\section{Discussion}\label{discussion}
\subsection{Open education}

\subsection{Future work}
\noindent
Cognition, affect / emotion, motivation, and metacognition \\
Experimentation and evaluation \\
Integration with virtual reality \\

\section{Curriculum}\label{curriculum}
%\subsection{Workshops}
Workshops

\section{Conclusions}\label{conclusion}

\appendix
\section{Building Tangible Landscape}

% -------------- BIBLIOGRAPHY --------------

% \bibliographystyle{elsarticle-num}
 \bibliographystyle{elsarticle-harv}
% \bibliographystyle{elsarticle-num-names}
% \bibliographystyle{model1a-num-names}
% \bibliographystyle{model1b-num-names}
% \bibliographystyle{model1c-num-names}
% \bibliographystyle{model1-num-names}
% \bibliographystyle{model2-names}
% \bibliographystyle{model3a-num-names}
% \bibliographystyle{model3-num-names}
% \bibliographystyle{model4-names}
% \bibliographystyle{model5-names}
% \bibliographystyle{model6-num-names}

\bibliography{tangible_teaching.bib}


\end{document}

%%
%% End of file `elsarticle-template-num.tex'.
